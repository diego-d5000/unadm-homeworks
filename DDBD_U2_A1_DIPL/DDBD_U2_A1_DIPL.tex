\documentclass[spanish,12pt,letterpapper]{article}
\usepackage{babel}
\usepackage[utf8]{inputenc}
\usepackage{graphicx}
\begin{document}
	\begin{titlepage}
		\begin{center}
			\includegraphics[width=0.6\textwidth]{../logoUnADM}~\\[1cm] 
			\textsc{Universidad Abierta y a Distancia de México}\\[0.8cm]
			\textsc{Desarrollo de Software}\\[1.8cm]
			
			\textbf{ \Large Actividad 1.- Modelo entidad-relación.}\\[3cm]
			
			Diego Antonio Plascencia Lara\\ ES1421004131 \\[0.4cm]
			Facilitador(a): CESAR ALEXIE CHAN PUC  \\
			Materia: Diseño de Bases de Datos\\
			Grupo: DS-DDBD-1601-B1-003 \\
			Unidad: II \\
			
			\vfill México D.F\\{\today}
			
		\end{center}
	\end{titlepage}
	\section{Diagrama entidad-relación}
	Antes de realizar el diagrama entidad-relación para esta actividad, decidí emplear una tabla, para primero identificar todas las entidad y atributos, para en el diagrama hacer las correspondencias de cardinalidad y conectar las entidades de una manera eficiente.
	
	\subsection{Identificación de atributos}
	\begin{center}
	\begin{tabular}{| p{4cm} | p{4cm} |}
	\hline
	
	\textbf{Entidad} & \textbf{Atributos}\\
	\hline
	Automóvil (E. débil) & numero de serie (pk)\linebreak marca \linebreak modelo \linebreak color \linebreak precio \linebreak placas\\
	\hline
	Cliente (E. fuerte) & id (pk) \linebreak numero de IFE \linebreak nombre \linebreak dirección \linebreak ciudad \linebreak numero de teléfono. \linebreak automovilSerie (fk) \\
	\hline
	Revisión (E. fuerte) & id (pk) \linebreak cambio filtro \linebreak cambio aceite \linebreak cambio frenos \linebreak cambio otros \linebreak automovilSerie (fk) \\
	
	\hline	
	\end{tabular}
	\end{center}
	
	\subsection{Diagrama}
		
	\begin{center}
	\includegraphics[width=1\textwidth]{./ermodel}~\\[1cm] 
	\end{center}
	
	\section{Conclusión}
	El diagrama de entidad-relación es importante por que te ayuda a identificar conceptualmente como organizar la base de datos de manera eficiente en base a los requerimientos y de la solución al problema, y al ser mas abstracto el diseño se puede implementar, cambiar, modificar, y optimizar mas fácilmente.\\
	
	Este diagrama es el mas utilizado y es análogo a un wireframe, lo que nos permite tener un prototipo de lo que será una base de datos. Y sin este prototipo probablemente sería mas tardado y difícil implementarla en un DBMS-
	
	\pagebreak
	\begin{thebibliography}{9}		
		\bibitem{rrhopkins} Robert J. Robbins. 
		\emph{Database Fundamentals}. Johns Hopkins University, [Disponible en: http://www.esp.org/db-fund.pdf].
		
		\bibitem{elmasriynavathe} Ramez Elmasri and Shamkant Navathe. 
		\emph{Fundamentals of Database Systems}. Pearson Education, [Disponible en: http://tinman.cs.gsu.edu/~raj/4710/f11/Ch01.pdf].

	\end{thebibliography}

\end{document}