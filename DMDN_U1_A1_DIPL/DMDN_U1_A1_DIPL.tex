\documentclass[spanish,12pt,letterpapper]{article}
\usepackage{babel}
\usepackage[utf8]{inputenc}
\usepackage{graphicx}
\begin{document}
	\begin{titlepage}
		\begin{center}
			\includegraphics[width=0.6\textwidth]{./logoUnADM}~\\[1cm] 
			\textsc{Universidad Abierta y a Distancia de México}\\[0.8cm]
			\textsc{Desarrollo de Software}\\[1.8cm]
			
			\textbf{ \Large Actividad 1. Importancia del modelado de negocios}\\[3cm]
			
			Diego Antonio Plascencia Lara\\ ES1421004131 \\[0.4cm]
			Facilitador(a): Neyda Lizzeth Moreno Cárdenas.\\
			Materia: Modelado de Negocios\\
			Grupo: DS-DMDN-1601-B1-007 \\
			Unidad: I \\
			
			\vfill México D.F\\{\today}
			
		\end{center}
	\end{titlepage}
	
	\section{Pre-concepto \\}
	El modelado de negocios es referente a la acción de crear un modelo de negocios, el cual es un plan, conceptualización o guía que tiene y sigue alguna organización o incluso una persona para llevar a cabo todo el proceso de negocio (costos, clientes, ingresos, producto, etc.) \\
	
	\section{Modelado de Negocios}
	\subsection{Concepto}
	Una negocio, es decir este sistema u organización que esta conformada por distintos grupos, tienen ciertas funciones, las cuales no son especificas de un solo departamento o grupo, sino que ``cruzan horizontalmente''. Y un modelo de negocios el cual ``describe las actividades clave de la organización y cómo se relacionan e interactúan con los recursos del negocio para lograr la meta establecida para el proceso'' (OMG,
	2008). ``El modelo debe concentrarse en las tareas y mecanismos clave principales del negocio'' \cite{panAdm}. \\
	
	\subsection{Características y elementos}
	
	\textbf{Lista las características principales del modelado de negocio, explicando que entiendes por cada una de ellas.\\}
	
	Proveen Información. \\
	Los modelos de negocios permiten conocer los mecanismos clave de un negocio, son base para sistemas de información, facilitan la identificación de ideas para mejorar el negocio.\\
	 
	Es la imagen de una empresa.\\
	permiten validar y experimentar conceptualmente con nuevas ideas y permiten presentar una nueva propuesta de trabajo para su financiamiento o algún otro objetivo.\\
	
	Para esta actividad me enfocare mas en el tema de las ``nuevas generaciones de modelos de negocio'' como lo es el llamado ``Business Model Canvas''. Este es un gráfico en el cual hay elementos que describen un producto, su infraestructura, clientes y finanzas, ayudando a las empresas a alinear sus actividades ilustrando las posibles compensaciones. Sus elementos son los siguientes: \\
	
	\begin{itemize}
		\item \textbf{Infraestructura} 
		\begin{itemize}
			\item Key Activities: Las actividades mas importantes en la ejecución de la propuesta de valor de una empresa. 
			\item Key Resources: Los recursos necesarios para crear valor para el cliente, son considerados activos para la empresa y necesarios para sostenerla.
			\item Partner Network: Para optimizar operaciones y reducir riesgos de un modelo de negocios, una organización debe cultivar sus relaciones comprador-proveedor para centrarse en sus actividades principales.  \\
		\end{itemize}
		\item \textbf{Oferta}
		\begin{itemize}
			\item Value Propositions: Los productos y servicios que ofrece un negocio para satisfacer las necesidades de sus clientes, estas deben distinguir al negocio de sus competidores. \\
		\end{itemize}
		\item \textbf{Clientes}
		\begin{itemize}
			\item Costumer Segments: Una empresa debe identificar los clientes a los que dirige el producto y segmentarlos en grupos basados en sus diferentes características y atributos, esto para implementar la apropiada estrategia y satisfacer las necesidades del grupo seleccionado.
			\item Channels: La compañía puede entregar sus propuestas de valor a sus clientes objetivo, a través de diferentes canales, estos deben ser rápidos, efectivos, eficientes y de costo efectivo. 
			\item Costumer Relationship: Para asegurar la sobrevivencia y éxito de cualquier negocio, la compañía debe identificar el tipo de relación que quiere crear con sus segmentos de clientes.
		\end{itemize}
		\item \textbf{Finanzas} 
		\begin{itemize}
			\item Cost Structure: Describe las mas importantes consecuencias monetarias mientras opera bajo diferentes modelos de negocio.
			\item Revenue Streams: El camino por el cual la compañía recibe ingresos por cada segmento de clientes. \\
		\end{itemize}
	\end{itemize}
	\subsection{Relación con el desarrollo de software}
	
	\textbf{¿Te has enterado de los problemas que ocasionan el desconocimiento o la falta de dominio de todos los requerimientos necesarios para el desarrollo de software?. ¿Qué pasaría si desconocieras los procesos en los que está implicado el software que desarrollarás? \\}
	
	Es importante tener conocimiento de el negocio y mas fácilmente si se tratan de nuevas propuestas para este como el presentado anteriormente, pues es un modelado dinámico que permite conocer de manera rápida y gráfica de que trata un negocio, poder recolectar casos de uso y entender el por que y que problema se esta resolviendo, que es lo que se quiere hacer y como se lleva a cabo. \\ \\
	De esta manera se desarrolla el software ``a la medida'' y no ``a ciegas'', sin embargo cabe aclarar que es importante conocer la importancia del dinamismo en el modelado de negocio y mas aun si se trata de una nueva propuesta en desarrollo, puesto que el iterar rápidamente permite moldear las propuestas de valor a lo que los clientes demandan, y poder tener éxito en el negocio. Ya que de no ser así podría ser tardado el desarrollo del software, por ello también existen ``las metodologías ágiles'' para el desarrollo de software.
	
	\pagebreak
	\begin{thebibliography}{9}
		\bibitem{algsRobert} Wikipedia, the free encyclopedia. 
		\emph{Business Model Canvas}. {[} Fecha de consulta: \today {]}. Disponible en: \textless https://en.wikipedia.org/wiki/Business\_Model\_Canvas \textgreater
		
		\bibitem{panAdm} León León, Oyuky María \& Asato España, Julio Armando. 
		\emph{La Importancia del Modelado de Procesos de
			Negocio como Herramienta para la Mejora e
			Innovación}. Panorama administrativo {[}en linea{]}, México. 2009, vol.4 num. 7  {[} Fecha de consulta: \today {]}. Disponible en: \textless http://132.248.9.34/hevila/Panoramaadministrativo/2009/no7/4.pdf \textgreater
	\end{thebibliography}
\end{document}