\documentclass[spanish,12pt,letterpapper]{article}
\usepackage{babel}
\usepackage[utf8]{inputenc}
\usepackage{graphicx}
\usepackage{hyperref}
\begin{document}
	\begin{titlepage}
		\begin{center}
			\includegraphics[width=0.6\textwidth]{../logoUnADM}~\\[1cm] 
			\textsc{Universidad Abierta y a Distancia de México}\\[0.8cm]
			\textsc{Desarrollo de Software}\\[1.8cm]
			
			\textbf{ \Large Actividad 2. Lectura de archivos}\\[3cm]
			
			Diego Antonio Plascencia Lara\\ ES1421004131 \\[0.4cm]
			Facilitador(a): LETICIA MARIA ANGELES GONZALEZ\\
			Asignatura: Programación orientada a objetos III\\
			Grupo: DS-DPO3-1502S-B2-002 \\
			Unidad: I \\
			
			\vfill México D.F\\{\today}
			
		\end{center}
	\end{titlepage}
	
	\section{Conclusiones}
	En esta actividad realice un programa cuya función es simple, leer un archivo de texto plano y mostrarlo en el programa, el proyecto se puede encontrar en el siguiente repositorio y para compilarlo y ejecutarlo es necesario contar con el Java 1.8:\\
	
	\url{https://github.com/diego-d5000/java-text-editor} \\
	
	En esta actividad aprendi como es la lógica, clases, paquetes y métodos que se utilizan para leer un archivo, y poder tener un flujo de entrada, para que información externa a el programa pueda ser leído y posteriormente utilizada.\\
	
	Esta base sirve para diferentes programas, es decir que es la forma en como obtienes información que anteriormente fue serializada y guardada en un archivo. El principio puede ser igual para diferentes tipos de datos entrantes y salientes, no necesariamente texto plano.
	
	
\end{document}