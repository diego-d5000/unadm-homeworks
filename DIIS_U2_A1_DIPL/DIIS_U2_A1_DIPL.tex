\documentclass[spanish,12pt,letterpapper]{article}
\usepackage{babel}
\usepackage[utf8]{inputenc}
\usepackage{graphicx}
\usepackage{hyperref}
\begin{document}
	\begin{titlepage}
		\begin{center}
			\includegraphics[width=0.6\textwidth]{../logoUnADM}~\\[1cm] 
			\textsc{Universidad Abierta y a Distancia de México}\\[0.8cm]
			\textsc{Desarrollo de Software}\\[1.8cm]
			
			\textbf{ \Large Actividad 1. Impacto de la Ingeniería de Software}\\[3cm]
			
			Diego Antonio Plascencia Lara\\ ES1421004131 \\[0.4cm]
			Facilitador(a): Martín Antonio Santos Romero\\
			Asignatura: Introducción a la Ingeniería de Software\\
			Grupo: DS-DIIS-1601-B2-006 \\
			Unidad: II \\
			
			\vfill México D.F\\{\today}
			
		\end{center}
	\end{titlepage}
	
	\section{Caso de estudio}
	Para esta actividad se plantea el caso de una empresa de manufactura de piel llamada Élite, el cual requiere un sistema para el control de las existencias de un almacén de materia prima. Se tiene un audio donde se puede escuchar la platica que tienen para recabar todos los requerimientos de el cliente en este caso Javier Mendoza. El audio se encuentra en el siguiente link:\\
	\url{https://unadmexico.blackboard.com/bbcswebdav/institution/DCEIT/2016_S1-B2/DS/03/DIIS/U2/Descargables/Empresa_manufacturera_de_producto_de_piel_elite.mp3}
	
	\section{Identificación de técnicas}
	Las técnicas que se usaron o se usaran para la recolección de requerimientos en el caso de uso fueron las siguientes:
	
	\paragraph{Entrevista.} En el audio se puede notar el uso de la entrevista, pues la charla con Javier Mendoza fue una charla de preguntas y respuestas, es decir Julieta Ramos y Fernando Robles entrevistaron a Javier Mendoza para saber detalles acerca de los procesos del negocio y de ciertas reglas de este, realizando preguntas abiertas y cerradas, pues no lo menciona pero hacen preguntas básicas acerca del negocio que probablemente fueron agendadas (cerradas), mientras que otras preguntas están basadas en las respuestas de las anteriores (abiertas).\\
	
	La entrevista es una técnica principal en la recolección donde el equipo de ingeniería de requerimientos pone una serie de preguntas acerca del sistema que ya tienen o el que se realizara, y los requerimientos se basan en estas respuestas. Existen dos tipos de entrevistas:
	\begin{itemize}
	\item \textbf{Cerradas:} Donde el interesado response una serie de preguntas predefinidas.
	\item \textbf{Abiertas:} No hay una agenda predefinida, el equipo de requerimientos explora una serie de temas con los interesados para entender mejor las necesidades
	\end{itemize}
	En la practica se usa una mezcla de ambas para ser mas preciso y comprender que es lo que los clientes hacen, como pueden interactuar con el sistema y las dificultades que tienen o podrían tener.
	
	\paragraph{Observación.} Se puede escuchar que Julieta Ramos menciona la necesidad de observar el proceso antes de entregar el análisis y propuesta, por lo que en un futuro se realizara la esta técnica.\\
	
	La observación es una técnica en la cual el analista esta presente en el lugar para poder tener noción de el entorno en donde se usara el proyecto, los procesos del negocio y la funcionalidad de la operación que se quiere construir en el software, ademas de que permite percatarse de el funcionamiento real que no se encuentra en los manuales de proceso.
	  
	
	
	\pagebreak
	\begin{thebibliography}{9}
	\bibitem{CompSciMag}  Maheshwari, Shikha. 
		\emph{A Comparative Analysis of Different types of Models in Software Development Life Cycle} (2012). International Journal of Advanced Research in Computer Science and Software Engineering. [Disponible en: http://www.ijarcsse.com/docs/papers/May2012/Volum2\_issue5/V2I500405.pdf ]
		
		\bibitem{sommerville}  Sommerville, Ian. 
		\emph{Software engineering} (2011). USA:  Pearson Education. 9th ed. 
		
		\bibitem{sommerville}  Fernández, Jorge. 
		\emph{Introducción a las metodologías ágiles: Otras formas de analizar y desarrolla}. FUOC. Fundación para la Universitat Oberta de Catalunya. 
		
	\end{thebibliography}
	
\end{document}