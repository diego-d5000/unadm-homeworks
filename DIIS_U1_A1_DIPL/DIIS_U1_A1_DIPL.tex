\documentclass[spanish,12pt,letterpapper]{article}
\usepackage{babel}
\usepackage[utf8]{inputenc}
\usepackage{graphicx}
\usepackage{hyperref}
\begin{document}
	\begin{titlepage}
		\begin{center}
			\includegraphics[width=0.6\textwidth]{../logoUnADM}~\\[1cm] 
			\textsc{Universidad Abierta y a Distancia de México}\\[0.8cm]
			\textsc{Desarrollo de Software}\\[1.8cm]
			
			\textbf{ \Large Actividad 1. Impacto de la Ingeniería de Software}\\[3cm]
			
			Diego Antonio Plascencia Lara\\ ES1421004131 \\[0.4cm]
			Facilitador(a): Martín Antonio Santos Romero\\
			Asignatura: Introducción a la Ingeniería de Software\\
			Grupo: DS-DIIS-1601-B2-006 \\
			Unidad: I \\
			
			\vfill México D.F\\{\today}
			
		\end{center}
	\end{titlepage}
	
	\section{Ingeniería de Software}
	\subsection{Definición}
	Es una disciplina que contempla todos los aspectos de la producción de software desde las especificaciones hasta el mantenimiento después del uso.\\
	
	El ingeniero de software debe aplicar las teorías, metodologías y herramientas apropiadas, incluso cuando estas no existan deben ser descubiertas teniendo en cuenta las restricciones organizacionales y financieras.\\
	
	La ingeniería de software no se limita al proceso técnico pues también debe tomar en cuenta la administración del proyecto y el desarrollo de herramientas, métodos y teorías para la producción del software.\\
	
	
	\subsection{Principales aplicaciones}
	
	\paragraph{Autónomas.} Son aplicaciones que no necesitan estar conectadas a la red, pues tienen todos los recursos necesarios, ejemplos de este tipo de aplicación son los procesadores de texto, CADs, manipulación de fotografiás.
	
	\paragraph{Interactivas basadas en transacciones.} Son aplicaciones que se ejecutan en una computadora remota y los usuarios acceden a ellas a través de sus computadoras, esto incluye las aplicaciones web, como comercios electrónicos, email, software de negocios.
	
	\paragraph{Control embebido.} Es software que controla y administra hardware, como celulares o microondas.
	
	\paragraph{Proceso por lotes.} Programas que ejecutan grandes lotes de información, con numerosas entras y salidas, ejemplos son los sistemas de facturación telefónica o programas de salarios.
	
	\paragraph{Entretenimiento.} Sistemas creados para uso personal y entretener a el usuario, como los videojuegos.
	
	\paragraph{Modelado y simulación.} Son desarrollados por científicos e ingenieros para modelar situaciones o procesos físicos.
	
	\paragraph{Recolección de datos.} Sirven para obtener datos de el ambiente ya sea de una maquina o de un lugar remoto, como el sistema de enfriamiento de una computadora.
	
	\paragraph{Sistemas de sistemas.} Son genéricos y están conformados por otros sistemas como las hojas de calculo.
	
	\pagebreak
	\begin{thebibliography}{9}
		\bibitem{sommerville}  Sommerville, Ian. 
		\emph{Software engineering} (2011). USA:  Pearson Education. 9th ed. 
		
	\end{thebibliography}
	
	
	
	
\end{document}