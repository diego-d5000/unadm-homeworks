\documentclass[spanish,12pt,letterpapper]{article}
\usepackage{babel}
\usepackage[utf8]{inputenc}
\usepackage{graphicx}
\usepackage{hyperref}
\begin{document}
	\begin{titlepage}
		\begin{center}
			\includegraphics[width=0.6\textwidth]{../logoUnADM}~\\[1cm] 
			\textsc{Universidad Abierta y a Distancia de México}\\[0.8cm]
			\textsc{Desarrollo de Software}\\[1.8cm]
			
			\textbf{ \Large Actividad 3. Diagrama de procesos del negocio}\\[3cm]
			
			Diego Antonio Plascencia Lara\\ ES1421004131 \\[0.4cm]
			Facilitador(a): Julia Alicia Reyes Rios\\
			Materia: Modelado de Negocios\\
			Grupo: DS-DMDN-1601-B1-007 \\
			Unidad: III \\
			
			\vfill México D.F\\{\today}
			
		\end{center}
	\end{titlepage}
	
	\section{Recapitulación}
	\subsection{Proceso a Diagramar}
	
	``EL ENTRENAMIENTO AUDITIVO INTERACTIVO''\\
	
	``El entrenamiento auditivo de un músico es una de las tareas más importantes para su vida profesional. En su etapa formativa es uno de los aspectos que deberían considerarse prioritarios, sin embargo, esto no necesariamente ocurre. Es una labor tediosa y difícil de llevar a cabo, especialmente en su etapa inicial. Para cubrir esta necesidad primordial, un grupo de maestros de asignaturas teórico musicales decidió plantear un subproyecto PAPIME para la implementación de un Laboratorio de Entrenamiento Auditivo Interactivo en la Escuela Nacional de Música, en donde los estudiantes pudieran acudir a trabajar de forma individual y utilizar programas computacionales que los ayudaran en su entrenamiento auditivo, mejorando la calidad de su formación musical''
	
	\subsection{Tareas}
	Proceso ``Uso del Laboratorio'':\\
	
	\textbf{ALUMNO:}
	
	Solicitar el Permiso de Uso del Recurso
	
	Usar Software Correspondiente al Objetivo
	
	Terminar la Practica y Abandonar el Aula\\
	
	\textbf{RECEPCIONISTA:}
	
	Otorgar Recursos
	
	Verificar el correcto uso de los recursos
	
	Registrar Salidas
	
	Dar mantenimiento al equipo de computo\\
	
	\textbf{DIRECCION GENERAL DE COMPUTO:}
	
	Atender Solicitudes de Reemplazo de Equipo
	
	\section{Diagrama}
	\begin{center}
	\includegraphics[width=1\textwidth]{./usecases}~\\[1cm]
	\end{center}
	
	En el diagrama se observan los 3 actores principales del proceso en la unidad anterior. Las actividades pasan a ser casos de uso y las sub-actividades a casos de uso con relación de inclusión, y las tareas en común extienden de una mas general.
	
	\pagebreak
	\begin{thebibliography}{9}
	\bibitem{maturanaModelamiento} Object Management Group. 
		\emph{Business Process Model and Notation}. {[} Fecha de consulta: \today {]}. Disponible en: \textless http://www.bpmn.org/ \textgreater	
	
		\bibitem{maturanaModelamiento} Maturana Ortiz, Jorge. 
		\emph{Modelamiento de Software y Negocios}. {[} Fecha de consulta: \today {]}. Disponible en: \textless http://www.info.univ-angers.fr/pub/maturana/files/Modelamiento\_de\_Software\_y\_Negocios.pdf \textgreater
		
		\bibitem{panAdm} León León, Oyuky María \& Asato España, Julio Armando. 
		\emph{La Importancia del Modelado de Procesos de
			Negocio como Herramienta para la Mejora e
			Innovación}. Panorama administrativo {[}en linea{]}, México. 2009, vol.4 num. 7  {[} Fecha de consulta: \today {]}. Disponible en: \textless http://132.248.9.34/hevila/Panoramaadministrativo/2009/no7/4.pdf \textgreater
	\end{thebibliography}
\end{document}