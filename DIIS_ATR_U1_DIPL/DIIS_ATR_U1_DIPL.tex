\documentclass[spanish,12pt,letterpapper]{article}
\usepackage{babel}
\usepackage[utf8]{inputenc}
\usepackage{graphicx}
\usepackage{hyperref}
\begin{document}
	\begin{titlepage}
		\begin{center}
			\includegraphics[width=0.6\textwidth]{../logoUnADM}~\\[1cm] 
			\textsc{Universidad Abierta y a Distancia de México}\\[0.8cm]
			\textsc{Desarrollo de Software}\\[1.8cm]
			
			\textbf{ \Large Autorreflexión Unidad 1}\\[3cm]
			
			Diego Antonio Plascencia Lara\\ ES1421004131 \\[0.4cm]
			Facilitador(a): Martín Antonio Santos Romero\\
			Asignatura: Introducción a la Ingeniería de Software\\
			Grupo: DS-DIIS-1601-B2-006 \\
			Unidad: I \\
			
			\vfill México D.F\\{\today}
			
		\end{center}
	\end{titlepage}
	
	\section{Preguntas}
	\paragraph{ Reflexiona la diferencia de un Ingeniero de Software y un Programador.\\}
	Un ingeniero de software es la persona que utiliza herramientas, técnicas, teorías y metodologías para llevar a cabo todos los procesos y pasos de la creación y producción de software, mientras que un programador es la persona técnica que a través de un lenguaje de programación escribe un programa que constituye el software.\\
	
Una analogía es la comparación entre un Arquitecto y un oficial de albañilería, el arquitecto se basa en teorías para planear que se tiene que hacer y cuanto va a costar, y un oficial en albañilería ejecuta el plan.\\
	
	\paragraph{ ¿Qué comentarios tienes sobre la importancia de la Ingeniería de Software en el desarrollo de un software de calidad?\\}
	Gracias a la Ingeniería de Software se puede no solo hacer software de calidad, sino integrarlo a tiempo, pues con la aplicación de los conocimientos científicos en en el campo de la computación (lo que hace la ingeniería de software) se puede hacer software optimo, rápido, eficiente, y que cumpla con ``estándares'' de calidad.
	
	\paragraph{¿Reflexiona sobre la diferencia de los métodos tradicionales contra los métodos ágiles de desarrollo de software? ¿En qué momento usarías uno y otro?\\}
	Todo depende del proyecto y la organización, normalmente (no siempre) los métodos ágiles se usan en empresas pequeñas o startups, donde la empresa aun esta definiendo un rumbo o vertiente, y los requerimientos de esta pueden cambiar repentinamente o puede haber desconocimiento de estos por parte del interesado.\\
	
	Mientras que las metodologías tradicionales se usan mas en organizaciones grandes que venden software especifico para empresas del mismo nivel, pues estas cuentan con un personal mayor, lo que dificulta ejecutar metodologías ágiles que requieren mas interacción entre el equipo, interesados, cliente, usuario, etc. y se sabe que estas permiten dar mas calidad a un producto final, sin embargo no permite los cambios.
	
	\paragraph{ Haz un análisis  a conciencia sobre lo que has aprendido hasta el momento, ¿consideras que alcanzaste la Competencia de Analizar los diferentes métodos de desarrollo de software para comprender cómo se relacionan con los elementos del ciclo de vida del software, identificando las características de cada método en un caso de estudio? ¿En qué temas debes reforzar y mejorar?\\}
	
	Si, considero que adquirí conocimientos y competencias suficientes (no bastas) sobre las metodologías mas usadas. Sin embargo pienso que debo reforzar en aprender y practicar mas específicamente alguna metodología, u observar como se efectúan estas en las empresas, pues la verdadera practica o ``aprender de los expertos'' me daría el conocimiento necesario para tener la competencia completa. 
	
	\pagebreak
	\begin{thebibliography}{9}
	\bibitem{CompSciMag}  Maheshwari, Shikha. 
		\emph{A Comparative Analysis of Different types of Models in Software Development Life Cycle} (2012). International Journal of Advanced Research in Computer Science and Software Engineering. [Disponible en: http://www.ijarcsse.com/docs/papers/May2012/Volum2\_issue5/V2I500405.pdf ]
		
		\bibitem{sommerville}  Sommerville, Ian. 
		\emph{Software engineering} (2011). USA:  Pearson Education. 9th ed. 
		
		\bibitem{fuoc}  Fernández, Jorge. 
		\emph{Introducción a las metodologías ágiles: Otras formas de analizar y desarrolla}. FUOC. Fundación para la Universitat Oberta de Catalunya. 
		
	\end{thebibliography}
	
\end{document}