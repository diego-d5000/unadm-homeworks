\documentclass[spanish,12pt,letterpapper]{article}
\usepackage{babel}
\usepackage[utf8]{inputenc}
\usepackage{graphicx}
\begin{document}
	\begin{titlepage}
		\begin{center}
			\includegraphics[width=0.6\textwidth]{./logoUnADM}~\\[1cm] 
			\textsc{Universidad Abierta y a Distancia de México}\\[0.8cm]
			\textsc{Desarrollo de Software}\\[1.8cm]
			
			\textbf{ \Large Actividad 1. Diseño de Bases de Datos  }\\[3cm]
			
			Diego Antonio Plascencia Lara\\ ES1421004131 \\[0.4cm]
			Facilitador(a): Sandra Ayanantzin Juárez Gutiérrez  \\
			Materia: Diseño de Bases de Datos\\
			Grupo: DS-DDBD-1502S-B1-002 \\
			Unidad: I \\
			
			\vfill México D.F\\{\today}
			
		\end{center}
	\end{titlepage}
	
	\section{Concepto de diseño de Bases de Datos y sus Características.\\}
	Existen dos tipos de diseño de base de datos. El diseño conceptual consiste en definir los elementos, interrelaciones de una base de datos, asi como información de las restricciones de los valor (``una regla que define los valores permisibles para un determinado dato''). Por otro lado el diseño físico de una base de datos, consiste en determinar la estructura física de una base de datos como que métodos de acceso serán utilizados y que indices incluir para mejorar el rendimiento. \\
	
	Los diseñadores de la BD deben trabajar en conjunto con los diferentes tipos de usuarios para formar las vistas (``es una porción restringida de la base de datos, también llamada vista de datos''), para que después sean integradas en un esquema completo (``una definición de la estructura lógica de la base de datos''). \\
	
	Para el diseño físico un equipo técnico debe conocer el SGBD, manejar el acceso a los datos y mejorar su eficiencia. ``Su objetivo es optimizar la combinación total de hardware, software y costo de los recursos humanos''\\
	
	\textbf{¿Qué aspectos deben tomarse en cuenta en el Diseño de una Bases de Datos?\\}
	Lo mas importante a tomar en cuenta para el diseño es al usuario, ya que este es el que le dará uso a los datos, por tal motivo la BD debe ajustarse al usuario y no de manera contraria.\\
	
	Otro punto a tomar en cuenta en el diseño son las normas y correcta lógica y estructura de los diferentes componentes, asi como sus relaciones entre los datos y como interactúan los usuarios con estos. 
	
	\pagebreak
	\begin{thebibliography}{9}
		\bibitem{garyBD} Hansen Gary, \& Hansen James. 
		\emph{Bases de Datos}. Prentice Hall, 2da edición.
	\end{thebibliography}

\end{document}