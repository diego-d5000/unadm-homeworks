\documentclass[spanish,12pt,letterpapper]{article}
\usepackage{babel}
\usepackage[utf8]{inputenc}
\usepackage{graphicx}
\begin{document}
	\begin{titlepage}
		\begin{center}
			\includegraphics[width=0.6\textwidth]{../logoUnADM}~\\[1cm] 
			\textsc{Universidad Abierta y a Distancia de México}\\[0.8cm]
			\textsc{Desarrollo de Software}\\[1.8cm]
			
			\textbf{ \Large Actividad 2. Modelo jerárquico y de red}\\[3cm]
			
			Diego Antonio Plascencia Lara\\ ES1421004131 \\[0.4cm]
			Facilitador(a): CESAR ALEXIE CHAN PUC  \\
			Materia: Diseño de Bases de Datos\\
			Grupo: DS-DDBD-1601-B1-003 \\
			Unidad: II \\
			
			\vfill México D.F\\{\today}
			
		\end{center}
	\end{titlepage}
	\section{Identificación de atributos}
	Antes de realizar el diagrama entidad-relación para esta actividad, decidí emplear una tabla, para primero identificar todas las entidad y atributos, para en el diagrama hacer las correspondencias de cardinalidad y conectar las entidades de una manera eficiente.
	
		\begin{center}
	\begin{tabular}{| p{4cm} | p{4cm} |}
	\hline
	
	\textbf{Entidad} & \textbf{Atributos}\\
	\hline
	Proveedor & clave\_proveedor \linebreak nombre \linebreak dirección \linebreak teléfono \linebreak pagina web\\
	\hline
	Cliente & clave\_cliente \linebreak nombre \linebreak dirección \linebreak teléfono\\
	\hline
	Producto & clave\_unica\_producto \linebreak nombre \linebreak precio actual \linebreak stock \linebreak nombre de proveedor \\
	\hline
	Categoría & clave\_categoria \linebreak nombre \linebreak descripción \\
	\hline
	Dirección & calle \linebreak numero \linebreak comuna \linebreak ciudad	\\	
	\hline
	Venta & clave\_venta \linebreak fecha \linebreak cliente \linebreak descuento \linebreak monto final \\
	\hline
	Teléfono & numero\\
		
	\hline	
	\end{tabular}
	\end{center}
	
	\section{Modelo jerárquico}
	
	\subsection{Conceptos}
	\paragraph{Segmento.} tiene campos (atributos), un segmento representa una entidad, y son nodos del árbol.
	\paragraph{Campo.} Es un atributo de la entidad.	
	\paragraph{N. Raíz.} Es el nodo inicial en el árbol y no tiene nodos padres, pero si nodos hijos.
	\paragraph{N. Padre.} Es un nodo que tiene otros subnodos (nodos hijos)
	\paragraph{N. Hijo.} Es un nodo que desciende de otro (nodo padre).\\
	
	\subsection{Diagrama}
	\begin{center}
	\includegraphics[width=0.8\textwidth]{./jerarquico}~\\[1cm]
	\end{center}
	\pagebreak
	
	\subsection{Explicación}
	
	En este diagrama se observan tres arboles, el primero que solo denota la relación 1:N entre categoría y producto, es decir una categoría tiene muchos productos y un producto solo tiene una categoría.\\
	
	El segundo árbol comienza con el Cliente el cual tiene teléfonos y direcciones (suponiendo que puede tener diferentes direcciones de entrega como casa o trabajo). Así como una venta tiene un solo cliente pero un cliente puede tener muchas ventas, las cuales puedes ser de varios productos.\\
	
	El tercer árbol tiene productos que distribuye, teléfonos y direcciones (diferentes puntos de distribución).\\
	\pagebreak
	
	\section{Modelo de red}
Una base de datos en red, deriva de la jerárquica pues esta trata de solucionar los problemas que tenía, como por ejemplo las relaciones N:M que no estaban permitidas en las jerárquicas, son permitidas en las de red mediante el uso de mas de un nodo padre.

Una base de datos en red es una colección de registros (records) y las relaciones de los registros están dadas por enlaces (links).
	
	\subsection{Conceptos}
	\paragraph{Registro (records).} Es análogo a una entidad en el modelo E-R, y es un nodo en la red.\\
	
	\paragraph{Campo (field).} Un campo esta dentro de un registro/nodo de la red, y este es análogo a un atributo de la entidad.\\
	
	\paragraph{Enlace (link).} Es una asociación entre 2 registros, y es como se indican las relaciones entre entidades.
	
	\subsection{Diagrama}
	
	\begin{center}
	\includegraphics[width=0.8\textwidth]{./jerarquico}~\\[1cm]
	\end{center}
	
	\subsection{Explicación}
	En este diagrama se observan los registros (Cliente, Venta, Proveedor, Dirección, Teléfono, Producto y Categoría) que aparecen solo una vez a diferencia del diagrama jerárquico, esto denota la omisión de repetir entidades/nodos o de tener mas de un conjunto, pues ahora un nodo puede tener mas de un nodo padre.\\
	
	El registro Teléfono y Dirección tienen dos nodos padres que son Cliente y Proveedor, pues tienen estos dos registros asociados en común. Después el registro Producto tiene tres nodos padres que son Categoría, Venta y Proveedor pues un producto tiene una categoría, esta presente en una venta y lo suministra un proveedor.
	
	
	\pagebreak
	\begin{thebibliography}{9}
	
	\bibitem{angelesmoraga} Baculima F. y Riofrio X. 
		\emph{EL MODELO DE DATOS JERÁRQUICO}. Universidad de Cuenca, [Disponible en: http://es.slideshare.net/FBaculima/modelo-jerarquico-y-modelo-de-red-de-base-de-datos].	

\bibitem{angelesmoraga} M. Ángeles Moraga de la Rubia. 
		\emph{EL MODELO DE DATOS JERÁRQUICO}. UNIVERSIDAD DE CASTILLA-LA MANCHA, [Disponible en: http://www.infcr.uclm.es/www/fruiz/bda/doc/trab/T0001\_MAMoraga.pdf].	
			
		\bibitem{rrhopkins} Robert J. Robbins. 
		\emph{Database Fundamentals}. Johns Hopkins University, [Disponible en: http://www.esp.org/db-fund.pdf].
		
		\bibitem{elmasriynavathe} Ramez Elmasri and Shamkant Navathe. 
		\emph{Fundamentals of Database Systems}. Pearson Education, [Disponible en: http://tinman.cs.gsu.edu/~raj/4710/f11/Ch01.pdf].

	\end{thebibliography}

\end{document}
