\documentclass[spanish,12pt,letterpapper]{article}
\usepackage{babel}
\usepackage[utf8]{inputenc}
\usepackage{graphicx}
\usepackage{hyperref}
\begin{document}
	\begin{titlepage}
		\begin{center}
			\includegraphics[width=0.6\textwidth]{../logoUnADM}~\\[1cm] 
			\textsc{Universidad Abierta y a Distancia de M\'exico}\\[0.8cm]
			\textsc{Desarrollo de Software}\\[1.8cm]
			
			\textbf{ \Large Actividad 3. Búsqueda y recorrido}\\[3cm]
			
			Diego Antonio Plascencia Lara\\ ES1421004131 \\[0.4cm]
			Facilitador(a): Alejandro Francisco Marquez Fuentes\\
			Asignatura: Estructura de datos\\
			Grupo: DS-DEDA-1502S-B1-002 \\
			Unidad: III \\
			
			\vfill M\'exico D.F\\{\today}
			
		\end{center}
	\end{titlepage}
	
	\paragraph{1. Investiga y define ¿qué es Recorrido en Arboles?\\}
	
	Se refiere a la acción de visitar, examinar, y/o actualizar los nodos de un árbol de manera sistemática, pasando por ellos una sola vez. Así mismo los tipos de recorrido se clasifican en el orden de visita.
	
	\paragraph{2. A partir de la información investigada, describe un ejemplo de\\}
	\subparagraph{Recorrido preorden. } Primero se visita la raíz, el sub árbol/ nodo izquierdo, luego el derecho. Es decir raíz, izquierda, derecha. 
	\subparagraph{Recorrido inorden. } En este recorrido primero se visitan los nodos/sub arboles izquierdos, luego la raíz y por ultimo los nodos/arboles derechos. Por lo tanto el recorrido es izquierda, raíz, derecha
	\subparagraph{Recorrido posorden. } Se comienza por el nodo/árbol izquierdo, posteriormente el derecho y por ultimo la raiz. Izquierda, derecha, raíz. 
	
	\paragraph{3. Finalmente, redacta una breve conclusión en torno al tema\\}
	Es importante conocer los métodos para recorrer un árbol, pues estos nos sirven para poder realizar el análisis, búsquedas, y tratamientos de los datos, parte fundamental de la materia, de las estructuras de datos. Esto en las ciencias de la computación y nuestra carrera tiene un gran impacto, ya que es lo que hace el software, tratar datos, información.

	\pagebreak
	\begin{thebibliography}{9}
		\bibitem{WikiWRecArbol} Wikipedia. 
		\emph{Recorrido en árboles}.  {[} Fecha de consulta: \today {]}. Disponible en: \textless 
		https://es.wikipedia.org/wiki/Recorrido\_de\_\%C3\%A1rboles \textgreater
		
		\bibitem{Goodrich} Goodrich, M. y Tamassia, R. (2010).
		\emph{Estructura de datos y algoritmos en Java}. México: CECSA.
		
		\bibitem{WikiWArbol} Wikipedia. 
		\emph{Árbol (informática)}.  {[} Fecha de consulta: \today {]}. Disponible en: \textless https://es.wikipedia.org/wiki/\%C3\%81rbol\_(inform\%C3\%A1tica) \textgreater
		
		\bibitem{WikiWArbol} Wikipedia. 
		\emph{Árbol binario}.  {[} Fecha de consulta: \today {]}. Disponible en: \textless https://es.wikipedia.org/wiki/\%C3\%81rbol\_binario \textgreater

	\end{thebibliography}
	
\end{document}