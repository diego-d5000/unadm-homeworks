\documentclass[spanish,12pt,letterpapper]{article}
\usepackage{babel}
\usepackage[utf8]{inputenc}
\usepackage{graphicx}
\usepackage{hyperref}
\begin{document}
	\begin{titlepage}
		\begin{center}
			\includegraphics[width=0.6\textwidth]{../logoUnADM}~\\[1cm] 
			\textsc{Universidad Abierta y a Distancia de México}\\[0.8cm]
			\textsc{Desarrollo de Software}\\[1.8cm]
			
			\textbf{ \Large Evidencia de aprendizaje. Metodología de desarrollo de software}\\[3cm]
			
			Diego Antonio Plascencia Lara\\ ES1421004131 \\[0.4cm]
			Facilitador(a): Martín Antonio Santos Romero\\
			Asignatura: Introducción a la Ingeniería de Software\\
			Grupo: DS-DIIS-1601-B2-006 \\
			Unidad: I \\
			
			\vfill México D.F\\{\today}
			
		\end{center}
	\end{titlepage}
	
	\section{Caso de Estudio}
	
	El Departamento de Recursos Materiales de la Universidad Abierta y a Distancia de México se encarga de proporcionar al personal los bienes materiales necesarios para ofrecer servicios de calidad a alumnos, docentes y público en general. Para llevar un mejor control de los bienes se ha solicitado la elaboración de un sistema de software para el manejo de algunas operaciones básicas. El principal requerimiento es que dicho sistema permite registrar la recepción de un bien en el almacén, asigne un número de inventario, y asigne el bien bajo resguardo del personal que lo ha solicitado, permitiendo imprimir un oficio de asignación bajo resguardo. La descripción que da el cliente, de cómo le gustaría que fuera es la siguiente: \\
	
	\begin{itemize}
    \item Que se mantenga un catálogo actualizado de todos los bienes que se pueden tener en la institución, proveedores y empleados. 

    \item Que se registre la entrada del bien al almacén (código, producto, marca, unidad de medida, cantidad, precio unitario, proveedor, factura). 

    \item Que el sistema muestre en pantalla la lista de bienes en almacén y permita seleccionar uno o más para inventariar, generando un número de inventario. 

    \item Que el sistema muestre los bienes inventariados sin asignar y permita seleccionar a un empleado y asignarle bajo resguardo dicho bien, generando el oficio de resguardo y registrando a quién se le asignó y cuál será su ubicación física. 

    \item Cuando los bienes ya queden inservibles por el tiempo de uso y deterioro, se permita darle de baja, generando un oficio de baja de la institución y baja de resguardo. 

    \item Que el sistema permita generar una lista de los bienes activos en dos periodos por año para su inventario físico, permitiendo marcar si el bien existe y capturar observaciones sobre su estado actual.   
    	\end{itemize}

Por los diversos compromisos del jefe del Departamento de Recursos Materiales, solamente puede tener entrevistas con los desarrolladores del software durante la primera semana del proyecto. El cliente necesita que haya resultados funcionales totales en máximo seis meses, donde al finalizar este periodo se tenga el sistema completo funcional, se implemente y se realicen las pruebas y se entregue un escrito con toda la documentación completa y detallada del desarrollo del sistema.

   
   \section{Resolución}
   \subsection{Características que el modelo debe cumplir}
   Las características principales que detecte en el caso de estudio fueron las siguientes:
   
   \begin{itemize}
   \item El interesado ya definió las funciones del sistema.
   \item No es un proyecto grande, es simple
   \item El interesado no esta disponible durante el proyecto, solo al inicio de este
   \item Hay un rango de tiempo limite de entrega
   \item Se requiere que la entrega sea única y final con el proyecto terminado.
   \item El proyecto requiere documentación extensa del proyecto.
   \end{itemize}
   
   \subsection{Identificación de opciones de modelos}
   
   Los 3 modelos que pueden aplicar son los siguientes:
   
   \begin{itemize}
   \item \textbf{Waterfall.} Un Modelo básico secuencial, esta enfocada a la entrega en un solo tiempo y enfatiza en la documentación y calendarizción.
   \item \textbf{V-Model.} Este modelo es similar a Waterfall, pues esta dirigido a proyectos que no tendran grandes cambios y se definen los requerimientos al principio, cada fase tiene su diseño de verificación y su verificación.
   \item \textbf{Iterative.} Este modelo es el menos acertado, inspirado en waterfall, este modelo hace pequeñas iteraciones de este modelo, evolucionando el software.
   \end{itemize}
   
   \textbf{N = no concuerda}
   
   \begin{tabular}{|p{3cm}|p{3cm}|p{3cm}|p{3cm}|}
   \hline
   \textbf{Característica del caso} & \textbf{Waterfall} & \textbf{V-Model} & \textbf{Iterative} \\
   \hline
   El interesado ya definió las funciones del sistema. & Todos los posibles requerimientos son capturados y documentados al inicio del proyecto & En la primera fase del proceso se deben de recabar todos los requerimientos y entenderlos claramente desde el punto de vista del cliente & N - Este modelo no intenta comenzar con las especificaciones completas del software.\\
   \hline
   No es un proyecto grande, es simple & El modelo es simple de implementar y adecuado para proyectos cortos & El modelo se inspira en waterfall, por lo que de igual manera es adecuado para proyectos cortos & N - Una de las desventajas conocidas de Iterative Model es que solo es aplicable a proyectos largos, debido a que es difícil particionar en iteraciones proyectos pequeños\\
   \hline
   El interesado no esta disponible durante el proyecto, solo al inicio de este. & Los requerimientos se definen al principio y carece de interacción con el cliente & De igual manera que el Waterfall todo se define al inicio & N - En las iteraciones hay interacción y redefinición de requerimientos con el cliente para evolucionar las pequeñas versiones del software \\
   \hline
   \end{tabular}
   
   \pagebreak

   \begin{tabular}{|p{3cm}|p{3cm}|p{3cm}|p{3cm}|}
   \hline
   \textbf{Característica del caso} & \textbf{Waterfall} & \textbf{V-Model} & \textbf{Iterative} \\
   \hline
   Se requiere que la entrega sea única y final con el proyecto terminado. & Busca la implementación del sistema en un solo tiempo & El código es la ultima fase, por lo que solo hay un entregable & N - En cada iteración un producto operacional es entregado\\
   \hline
   El proyecto requiere documentación extensa del proyecto. & Hace una especial énfasis en la documentación extensa. & N - Especifica que los test deben ser documentados, pero no menciona que la documentación en general deba extensa & N - no hace mención al respecto\\
      \hline
Hay un rango de tiempo limite de entrega & Enfatiza en la planeación, calendarizacion y tiempos de entrega. & N - No tiene una especificación exacta en este punto & Este modelo lleva menos tiempo en la operativa inicial, se toma en cuenta el tiempo de entrega para hacer la fragmentación en iteraciones, se permite hacer mas de una iteración al mismo tiempo. \\
   \hline
   \end{tabular}
   
   \subsection{Elección del modelo}
   El modelo mas acertado es el Waterfall, ya que es un modelo básico y lineal que toma los requerimientos como entrada y da como salida el software según estos, es bueno para proyectos cortos y no necesita de mucha interacción con el cliente, que es el punto clave de el caso de uso, así como la documentación. De manera contraria el menos acertado es el Iterative Model, ya que este si requiere que el cliente este presente y tome un papel activo en el proyecto.
   
   \section{Conclusiones}
   Es importante saber elegir la metodología o modelo correcto, ya que no debemos de tomar una sola para todos los proyectos, pues la elección del modelo y la gestión del proyecto impacta en muchas variables de este como el tiempo, costos y optimización.\\
   
   Esta actividad de igual manera sirvió para reforzar los conocimientos de los modelos y hacer comparativas, cosa que debe hacerse al inicio de un proyecto. Pues hacer la mejor elección impacta directamente en la calidad de nuestro software.

	
	\pagebreak
	\begin{thebibliography}{9}
	\bibitem{CompSciMag}  Maheshwari, Shikha. 
		\emph{A Comparative Analysis of Different types of Models in Software Development Life Cycle} (2012). International Journal of Advanced Research in Computer Science and Software Engineering. [Disponible en: http://www.ijarcsse.com/docs/papers/May2012/Volum2\_issue5/V2I500405.pdf ]
		
		\bibitem{sommerville}  Sommerville, Ian. 
		\emph{Software engineering} (2011). USA:  Pearson Education. 9th ed. 
		
		\bibitem{fuoc}  Fernández, Jorge. 
		\emph{Introducción a las metodologías ágiles: Otras formas de analizar y desarrolla}. FUOC. Fundación para la Universitat Oberta de Catalunya. 
		
	\end{thebibliography}
	
\end{document}