\documentclass[spanish,12pt,letterpapper]{article}
\usepackage{babel}
\usepackage[utf8]{inputenc}
\usepackage{graphicx}
\usepackage{hyperref}
\begin{document}
	\begin{titlepage}
		\begin{center}
			\includegraphics[width=0.6\textwidth]{../logoUnADM}~\\[1cm] 
			\textsc{Universidad Abierta y a Distancia de México}\\[0.8cm]
			\textsc{Desarrollo de Software}\\[1.8cm]
			
			\textbf{ \Large Evidencia de aprendizaje. Metodología de desarrollo de software}\\[3cm]
			
			Diego Antonio Plascencia Lara\\ ES1421004131 \\[0.4cm]
			Facilitador(a): Martín Antonio Santos Romero\\
			Asignatura: Introducción a la Ingeniería de Software\\
			Grupo: DS-DIIS-1601-B2-006 \\
			Unidad: I \\
			
			\vfill México D.F\\{\today}
			
		\end{center}
	\end{titlepage}
	
	\section{Caso de Estudio}
	
	El Departamento de Recursos Materiales de la Universidad Abierta y a Distancia de México se encarga de proporcionar al personal los bienes materiales necesarios para ofrecer servicios de calidad a alumnos, docentes y público en general. Para llevar un mejor control de los bienes se ha solicitado la elaboración de un sistema de software para el manejo de algunas operaciones básicas. El principal requerimiento es que dicho sistema permite registrar la recepción de un bien en el almacén, asigne un número de inventario, y asigne el bien bajo resguardo del personal que lo ha solicitado, permitiendo imprimir un oficio de asignación bajo resguardo. La descripción que da el cliente, de cómo le gustaría que fuera es la siguiente: \\
	
	\begin{itemize}
    \item Que se mantenga un catálogo actualizado de todos los bienes que se pueden tener en la institución, proveedores y empleados. 

    \item Que se registre la entrada del bien al almacén (código, producto, marca, unidad de medida, cantidad, precio unitario, proveedor, factura). 

    \item Que el sistema muestre en pantalla la lista de bienes en almacén y permita seleccionar uno o más para inventariar, generando un número de inventario. 

    \item Que el sistema muestre los bienes inventariados sin asignar y permita seleccionar a un empleado y asignarle bajo resguardo dicho bien, generando el oficio de resguardo y registrando a quién se le asignó y cuál será su ubicación física. 

    \item Cuando los bienes ya queden inservibles por el tiempo de uso y deterioro, se permita darle de baja, generando un oficio de baja de la institución y baja de resguardo. 

    \item Que el sistema permita generar una lista de los bienes activos en dos periodos por año para su inventario físico, permitiendo marcar si el bien existe y capturar observaciones sobre su estado actual.   
    	\end{itemize}

Por los diversos compromisos del jefe del Departamento de Recursos Materiales, solamente puede tener entrevistas con los desarrolladores del software durante la primera semana del proyecto. El cliente necesita que haya resultados funcionales totales en máximo seis meses, donde al finalizar este periodo se tenga el sistema completo funcional, se implemente y se realicen las pruebas y se entregue un escrito con toda la documentación completa y detallada del desarrollo del sistema.

   
   \section{Resolución}
   \subsection{Características que el modelo debe cumplir}
   Las características principales que detecte en el caso de estudio fueron las siguientes:
   
   \begin{itemize}
   \item El interesado ya definió las funciones del sistema.
   \item No es un proyecto grande, es simple
   \item El interesado no esta disponible durante el proyecto, solo al inicio de este
   \item Hay un rango de tiempo limite de entrega
   \item Se requiere que la entrega sea única y final con el proyecto terminado.
   \item El proyecto requiere documentación extensa del proyecto.
   \end{itemize}
   
   \subsection{Identificación de opciones de modelos}
   
   Los 3 modelos que pueden aplicar son los siguientes:
   
   \begin{itemize}
   \item \textbf{Modelo 1.} Por que...
   \item \textbf{Modelo 2.} Por que...
   \item \textbf{Modelo 3.} Por que...
   \end{itemize}
   
   \begin{tabular}{|p{3cm}|p{3cm}|p{3cm}|p{3cm}|}
   \hline
   \textbf{Característica del caso} & \textbf{Modelo 1} & \textbf{Modelo 2} & \textbf{Modelo 3} \\
   \hline
   \end{tabular}
   
   \subsection{Elección del modelo}
   Decidí usar X por que...
   
   \section{Conclusiones}
   Es importante...

	
	\pagebreak
	\begin{thebibliography}{9}
	\bibitem{CompSciMag}  Maheshwari, Shikha. 
		\emph{A Comparative Analysis of Different types of Models in Software Development Life Cycle} (2012). International Journal of Advanced Research in Computer Science and Software Engineering. [Disponible en: http://www.ijarcsse.com/docs/papers/May2012/Volum2\_issue5/V2I500405.pdf ]
		
		\bibitem{sommerville}  Sommerville, Ian. 
		\emph{Software engineering} (2011). USA:  Pearson Education. 9th ed. 
		
		\bibitem{fuoc}  Fernández, Jorge. 
		\emph{Introducción a las metodologías ágiles: Otras formas de analizar y desarrolla}. FUOC. Fundación para la Universitat Oberta de Catalunya. 
		
	\end{thebibliography}
	
\end{document}