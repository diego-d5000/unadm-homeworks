\documentclass[spanish,12pt,letterpapper]{article}
\usepackage{babel}
\usepackage[utf8]{inputenc}
\usepackage{graphicx}
\usepackage{hyperref}
\begin{document}
	\begin{titlepage}
		\begin{center}
			\includegraphics[width=0.6\textwidth]{../logoUnADM}~\\[1cm] 
			\textsc{Universidad Abierta y a Distancia de México}\\[0.8cm]
			\textsc{Desarrollo de Software}\\[1.8cm]
			
			\textbf{ \Large Actividad 2. Componentes de un Modelo de Negocios }\\[3cm]
			
			Diego Antonio Plascencia Lara\\ ES1421004131 \\[0.4cm]
			Facilitador(a): Julia Alicia Reyes Rios\\
			Materia: Modelado de Negocios\\
			Grupo: DS-DMDN-1601-B1-007 \\
			Unidad: IV \\
			
			\vfill México D.F\\{\today}
			
		\end{center}
	\end{titlepage}
	
	\section{Caso de estudio}
	Para esta actividad he decido elegir un caso de estudio genérico, una abacería o tienda común moderna. Este caso tiene un flujo básico que comienza con el proveedor y termina con el cliente, por lo que los actores, sus actividades y sub-actividades desglosadas quedarían de la siguiente manera:\\
	
	\textbf{PROVEEDOR:\\}
	Elabora los productos\\
	Abastece los productos que vende a la tienda.\\	
	
	\textbf{CLIENTE:\\}
	Selecciona el producto.
	\begin{itemize}
	\item ve la lista de productos por las interfaces tecnológicas o acude a la tienda física
	\item selecciona los productos de su agrado
	\item agrega los productos a su "carrito" físico o digital\\
	\end{itemize}
	Paga la compra.
	\begin{itemize}
	\item revisa los productos que ha decidido comprar
	\item si decide pagar con tarjeta la entrega para ser deslizada, sino entrega en efectivo
	\item recibe un ticket de compra.\\
	\end{itemize}
	
	\textbf{GERENTE:\\}
	Realiza un control del inventario.
	\begin{itemize}
	\item recibe los productos del proveedor
	\item verifica que estén en correcto estado
	\item actualiza el "stock" en el sistema\\
	\end{itemize}
	Verifica las entradas de dinero.
	\begin{itemize}
	\item pide al sistema un reporte del total de ventas
	\item verifica si coincide con el dinero en caja\\
	\end{itemize}
	Aplica estrategias de marketing.
	\begin{itemize}
	\item pide reportes al sistema de los productos mas vendidos y los menos vendidos
	\item dependiendo de la zona y el inventario, anuncia promociones a los clientes\\
	\end{itemize}
	
	\textbf{VENDEDOR:\\}
	Da mantenimiento a la tienda.
	\begin{itemize}
	\item verifica que el sistema de computo funcione
	\item hace la limpieza del lugar
	\item acomoda los productos
	\item pone el precio de cada producto en el mostrador\\
	\end{itemize}
	Atiende solicitudes de compra de los clientes.
	\begin{itemize}
	\item ayuda a encontrar productos
	\item ayuda al proceso de compra\\
	\end{itemize}
	Recibe los pagos.
	\begin{itemize}
	\item revisa los productos que el cliente lleva
	\item si decide pagar con tarjeta se le solicita al cliente para deslizarla, sino se pide el efectivo y si entrega el cambio si lo hay
	\item entrega el ticket de compra\\
	\end{itemize}
	
	\textbf{REPARTIDOR:\\}
	Mantiene el vehículo en buen estado.
	\begin{itemize}
	\item se cerciora de que su equipo de seguridad este completo
	\item verifica que el vehículo funcione y este en optimas condiciones\\
	\end{itemize}
	Atiende solicitudes de pedidos a domicilio.
	\begin{itemize}
	\item verifica en el sistema que no haya pedidos pendientes en la zona
	\item si hay pedidos se traza una ruta y estimación de entrega
	\item se toman los productos de la tienda solicitados por los clientes en linea\\
	\end{itemize}
	Entrega productos al domicilio del cliente.\\
	Recibe los pagos.
	\begin{itemize}
	\item revisa los productos que el cliente lleva
	\item si decide pagar con tarjeta se le solicita al cliente para deslizarla, sino se pide el efectivo y si entrega el cambio si lo hay
	\item entrega el ticket de compra\\
	\end{itemize}
	
	
	\section{Modelo de Recursos}
	\begin{tabular}{c c}
	\hline
	\textbf{Recursos Humanos} & \textbf{Recursos Materiales, Físicos, Datos}\\
	\hline
	Proveedor (externo) & Pedido (datos)\\
	Cliente (externo) & Compras (datos)\\
	Vendedor & Ubicaciones (datos)\\
	Gerente & Clientes (datos)\\
	Repartidor & Sistema de inventarios (datos)\\
	& Sistema de compras (datos)\\
	& Productos (material)\\
	& Anaqueles (físico)\\
	& Tablets (físico)\\
	& Computadoras (físico)\\
	\hline
	\end{tabular}
	
	\section{Modelo de Metas}
	Para modelar las metas utilizare el diagrama "Use Cases" de UML, ya que este también sirve para indicar las metas, o a lo que se pretende llegar para cada actor.
	\begin{center}
	\includegraphics[width=0.7\textwidth]{./UC}~\\[1cm] 
	\end{center}
	
	\section{Atributos y Relaciones}
	Para modelar los atributos y relaciones utilizare un diagrama relacional, este posteriormente puede servir para integrar una base de datos.
	\begin{center}
	\includegraphics[width=0.7\textwidth]{./rel}~\\[1cm] 
	\end{center}
	
	\pagebreak
	\begin{thebibliography}{9}
	\bibitem{Alexander, Ian. Beus-Dukic, Ljerka} . 
		\emph{Discovering Requirements: How to Specify Products snd Services}. England. Wiley
	
	\bibitem{maturanaModelamiento} Object Management Group. 
		\emph{Business Process Model and Notation}. {[} Fecha de consulta: \today {]}. Disponible en: \textless http://www.bpmn.org/ \textgreater	
	
		\bibitem{maturanaModelamiento} Maturana Ortiz, Jorge. 
		\emph{Modelamiento de Software y Negocios}. {[} Fecha de consulta: \today {]}. Disponible en: \textless http://www.info.univ-angers.fr/pub/maturana/files/Modelamiento\_de\_Software\_y\_Negocios.pdf \textgreater
		
		\bibitem{panAdm} León León, Oyuky María \& Asato España, Julio Armando. 
		\emph{La Importancia del Modelado de Procesos de
			Negocio como Herramienta para la Mejora e
			Innovación}. Panorama administrativo {[}en linea{]}, México. 2009, vol.4 num. 7  {[} Fecha de consulta: \today {]}. Disponible en: \textless http://132.248.9.34/hevila/Panoramaadministrativo/2009/no7/4.pdf \textgreater
	\end{thebibliography}
\end{document}