\documentclass[spanish,12pt,letterpapper]{article}
\usepackage{graphicx}
\usepackage{babel}
\begin{document}
	\begin{titlepage}
		\begin{center}
			\includegraphics[width=0.6\textwidth]{./logoUnADM}~\\[1cm] 
			\textsc{Universidad Abierta y a Distancia de M\'exico}\\[0.8cm]
			\textsc{Desarrollo de Software}\\[1.8cm]
			
			\textbf{ \Large Actividad 1. Relaci\'on de algoritmos y estructuras de datos}\\[3cm]
			
			Diego Antonio Plascencia Lara\\ ES1421004131 \\[0.4cm]
			Facilitador(a): Alejandro Francisco Marquez Fuentes\\
			Asignatura: Estructura de datos\\
			Grupo: DS-DEDA-1502S-B1-002 \\
			Unidad: I \\
			
			\vfill M\'exico D.F\\{\today}
			
		\end{center}
	\end{titlepage}
	
	\section{Definiciones \\[1cm]}
	
	\paragraph{Algoritmo.}
	M\'etodos para resolver problemas que son adecuados para una implementaci\'on en computadora. En las ciencias de la computaci\'on es descrito como una serie de pasos o m\'etodo finito, determinado y efectivo para resolver un problema y que pueden ser implementados en un programa computacional.
	
	\paragraph{Queue.}
	Una cola FIFO o cola simplemente, es una colecci\'on basada en la pol\'itica de primeras entradas - primeras salidas (first-in first-out). Esta pol\'itica de hacer las tareas en el mismo orden en el que llegan lo encontramos frecuentemente: en la fila para el teatro, en las casetas de coches, las tareas esperando a ser servidas por una aplicaci\'on en tu computadora.
	
	\paragraph{Stack.}
	Esta basada en la pol\'itica ultimas entradas, primeras salidas o LIFO (last-in first-out). Un ejemplo sencillo es el correo electr\'onico, cuando te llegan los correos se apilan, y el ultimo que te llego es el que se encuentra al principio, y es el que lees primero.
	
	\paragraph{List.}
	En esta estructura de datos los valores no tienen que ser almacenados adyacentemente en la memoria (como en un array), sino que unicamente se requiere que los valores tengan un apuntador adicional que indique en que parte de la memoria se encuentra siguiente valor, por lo que para usar una lista solo debemos conocer donde esta el primer valor. Un ejemplo es una "To-Do List" que es una lista donde normalmente podemos realizar la tarea que deseemos, pero tambi\'en es com\'un que para llegar a una tarea se tengan que realizar las anteriores. 
	
	\pagebreak
	
	\section{Relaci\'on entre los algoritmos y las estructuras de datos \\[1cm]}
	La relaci\'on que existe entre un algoritmo (no necesariamente computacional) y las estructuras de datos no es directa, ya uno puede existir sin la necesidad del otro (dependiendo del contexto), pero directamente en las ciencias de la computaci\'on e implicitamente en la cotidianidad estas van ligadas. Esto por que normalmente un algoritmo necesita conocimiento (datos) para poder funcionar, puesto que a partir de la informaci\'on esta puede ser tratada para llegar finalmente a otro dato, o realizar alguna acci\'on gracias a estos. \\
	
	En palabras sencillas, un algoritmo necesita de la manipulaci\'on de estructuras de datos para poder funcionar, y los datos son in\'utiles sin un proceso que les de su utilidad. Por ejemplo si quisi\'eramos resolver un simple problema de pelar un pl\'atano, esto implica tener que conocer datos acerca de nosotros mismos, el objeto al que le realizaremos la acci\'on, medidas, ambiente, esto le da una utilidad a dichos datos. 
	
	\pagebreak
	
	\begin{thebibliography}{9}
		\bibitem{algsRobert} Sedgewick R. and Wayne K. 
		\emph{Algorithms}. Princeton University, Princeton, 4th edition, 2011.
		
		\bibitem{DSCortina} Cortina Tom. 
		\emph{Principles of Computing}. Camegie Mellon University, Pittsburgh,  {[} Fecha de consulta: 24 de Julio del 2015 {]}. Disponible en: \textless http://www.cs.cmu.edu/~tcortina/15110sp12/Unit06PtA.pdf \textgreater
	\end{thebibliography}
	
\end{document}