\documentclass[spanish,12pt,letterpapper]{article}
\usepackage{babel}
\usepackage[utf8]{inputenc}
\usepackage{graphicx}
\usepackage{hyperref}
\begin{document}
	\begin{titlepage}
		\begin{center}
			\includegraphics[width=0.6\textwidth]{../logoUnADM}~\\[1cm] 
			\textsc{Universidad Abierta y a Distancia de México}\\[0.8cm]
			\textsc{Desarrollo de Software}\\[1.8cm]
			
			\textbf{ \Large Evidencia de aprendizaje. Programa de archivos}\\[3cm]
			
			Diego Antonio Plascencia Lara\\ ES1421004131 \\[0.4cm]
			Facilitador(a): LETICIA MARIA ANGELES GONZALEZ\\
			Asignatura: Programación orientada a objetos III\\
			Grupo: DS-DPO3-1502S-B2-002 \\
			Unidad: I \\
			
			\vfill México D.F\\{\today}
			
		\end{center}
	\end{titlepage}
	
	\section{Conclusiones}
	Para esta actividad continué la anterior conservando la lógica de la apertura de archivos pero cambiando totalmente lo demás, por lo que para esta evidencia cambie Swing por JavaFx, y encapsule la lógica y manejo de archivos en una clase, realizando los métodos correspondientes, como se implementaría en una clase de la capa de abstracción con sus transacciones "CRUD", para que el modelo del documento y la UI convivieran utilice el patrón MVC.\\
	
	El proyecto se encuentra en el siguiente repositorio y para compilarlo y ejecutarlo es necesario contar con Java 1.8:\\
	
	\url{https://github.com/diego-d5000/java-text-editor} \\
	
\end{document}