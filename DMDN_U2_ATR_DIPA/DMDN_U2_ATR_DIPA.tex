\documentclass[spanish,12pt,letterpapper]{article}
\usepackage{babel}
\usepackage[utf8]{inputenc}
\usepackage{graphicx}
\usepackage{hyperref}
\begin{document}
	\begin{titlepage}
		\begin{center}
			\includegraphics[width=0.6\textwidth]{../logoUnADM}~\\[1cm] 
			\textsc{Universidad Abierta y a Distancia de México}\\[0.8cm]
			\textsc{Desarrollo de Software}\\[1.8cm]
			
			\textbf{ \Large Autorreflexión Unidad 2}\\[3cm]
			
			Diego Antonio Plascencia Lara\\ ES1421004131 \\[0.4cm]
			Facilitador(a): Julia Alicia Reyes Rios\\
			Materia: Modelado de Negocios\\
			Grupo: DS-DMDN-1601-B1-007 \\
			Unidad: II \\
			
			\vfill México D.F\\{\today}
			
		\end{center}
	\end{titlepage}
	
	\textbf{¿Que aprendí al elaborar esta evidencia de aprendizaje?\\}
	
	En esta evidencia aprendí el completo proceso de documentación y modelado de un negocio usando BPMN. Aprendí a identificar tareas y actores en un requerimiento, a enlazar los objetos, conexiones y artefactos del BPMN con los requerimientos y finalmente a elaborar el manual de un negocio.\\
	
	Aprendí que es importante conocer el modelado de negocios y a utilizar BPMN, pues en mi opinión este tipo de modelado es mas dirigido a negocios en general, mas simple, minimalista y menos saturado que el UML, pues es mas fácil de entender e interpretar, a pesar de tener similitudes con este\\.
	
	El conocimiento general que obtuve fue el uso de BPMN para modelar negocios.
	
	
	
	\pagebreak
	\begin{thebibliography}{9}
	\bibitem{maturanaModelamiento} Object Management Group. 
		\emph{Business Process Model and Notation}. {[} Fecha de consulta: \today {]}. Disponible en: \textless http://www.bpmn.org/ \textgreater	
	\end{thebibliography}
\end{document}