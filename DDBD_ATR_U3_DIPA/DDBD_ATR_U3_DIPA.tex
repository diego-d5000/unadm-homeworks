\documentclass[spanish,12pt,letterpapper]{article}
\usepackage{babel}
\usepackage[utf8]{inputenc}
\usepackage{graphicx}
\begin{document}
	\begin{titlepage}
		\begin{center}
			\includegraphics[width=0.6\textwidth]{../logoUnADM}~\\[1cm] 
			\textsc{Universidad Abierta y a Distancia de México}\\[0.8cm]
			\textsc{Desarrollo de Software}\\[1.8cm]
			
			\textbf{ \Large Autorreflexión Unidad 3}\\[3cm]
			
			Diego Antonio Plascencia Lara\\ ES1421004131 \\[0.4cm]
			Facilitador(a): CESAR ALEXIE CHAN PUC  \\
			Materia: Diseño de Bases de Datos\\
			Grupo: DS-DDBD-1601-B1-003 \\
			Unidad: III \\
			
			\vfill México D.F\\{\today}
			
		\end{center}
	\end{titlepage}
	
	\begin{itemize}
	
\item \textbf{¿Cuál es el objetivo del álgebra relacional?\\}
El de denotar con un lenguaje teórico, el álgebra y parte de teoría de conjuntos, las operaciones básicas para la obtención de datos de una DB.\\
	
\item \textbf{¿Cuál es el objetivo del cálculo relacional?\\}
El mismo objetivo del álgebra relacional, solo que esta no es procedimental, es de tipo declarativo, es decir solo indica que se desea obtener.\\

\item \textbf{En una tabla menciona las principales sentencias de DDL.\\}
\begin{center}
\begin{tabular}{|c|c|c|}
\hline
\textbf{Sentencia} & \textbf{Función}\\
\hline
CREATE & crea nuevas tablas.\\
\hline
DROP & elimina tablas.\\
\hline
ALTER & modifica las tablas (campos y/o definiciones).\\
\hline
\end{tabular}
\end{center}

\item \textbf{En una tabla menciona las principales sentencias de DML\\}
\begin{center}
\begin{tabular}{|c|c|c|}
\hline
\textbf{Sentencia} & \textbf{Función}\\
\hline
SELECT & consulta los registros.\\
\hline
INSERT & introduce nuevos registros.\\
\hline
UPDATE & modifica los datos en los registros\\
\hline
DELETE & elimina registros\\
\hline
\end{tabular}
\end{center}

\item \textbf{Diferencia entre DDL y DML\\}
el DDL sirve para definir el esquema (tablas) de los registros, mientras que el DML sirve para manipular registros que se basaron en estos esquemas.\\

\item \textbf{¿Qué es una vista?\\}
Encapsula una serie de instrucciones SQL regresando una supuesta nueva tabla para que se consulte directamente a esta evitando mostrar otras con datos sensibles implicadas.\\

	\end{itemize}
	
	\pagebreak
	\begin{thebibliography}{9}
		
		\bibitem{rrhopkins} Robert J. Robbins. 
		\emph{Database Fundamentals}. Johns Hopkins University, [Disponible en: http://www.esp.org/db-fund.pdf].
		
		\bibitem{elmasriynavathe} Ramez Elmasri and Shamkant Navathe. 
		\emph{Fundamentals of Database Systems}. Pearson Education, [Disponible en: http://tinman.cs.gsu.edu/~raj/4710/f11/Ch01.pdf].
		
		\bibitem{psqldocs} The PostgreSQL Development Team. 
		\emph{PostgreSQL 7.0 Docs}. PostgreSQL, [Disponible en: http://www.postgresql.org/docs/7.0/static/postgres.htm].
		

	\end{thebibliography}

\end{document}