\documentclass[spanish,12pt,letterpapper]{article}
\usepackage{babel}
\usepackage[utf8]{inputenc}
\usepackage{graphicx}
\usepackage{hyperref}
\begin{document}
	\begin{titlepage}
		\begin{center}
			\includegraphics[width=0.6\textwidth]{../logoUnADM}~\\[1cm] 
			\textsc{Universidad Abierta y a Distancia de M\'exico}\\[0.8cm]
			\textsc{Desarrollo de Software}\\[1.8cm]
			
			\textbf{ \Large Evidencia de aprendizaje. Métodos de ordenación y búsqueda}\\[3cm]
			
			Diego Antonio Plascencia Lara\\ ES1421004131 \\[0.4cm]
			Facilitador(a): Alejandro Francisco Marquez Fuentes\\
			Asignatura: Estructura de datos\\
			Grupo: DS-DEDA-1502S-B1-002 \\
			Unidad: II \\
			
			\vfill M\'exico D.F\\{\today}
			
		\end{center}
	\end{titlepage}
	
	\section{Contador de caracteres \\}
	
	Normalmente la mayoría de los lenguajes utilizados hoy en día, traen funciones built-in que prácticamente ya hacen el objetivo de esta practica ``contar caracteres'', o al menos funciones que hacen mas fácil realizarlo.\\
	
	El siguiente link lleva hacia un repositorio lleno de algoritmos propios, varios de ellos son de ordenamiento, para esta actividad el archivo correspondiente es ``char\_counter.py'':\\
	
	\url{https://github.com/diego-d5000/algorithms}
	
	\section{Conclusiones \\}
	\textbf{¿Cómo lo resolverías y que utilidad le encontraras?\\}
	
	Haciendo uso de las expresiones regulares (presentes en la mayoría de los lenguajes), separar las palabras en arreglos, para después contar los caracteres de cada una de ellas por medio de la iteración sobre la cadena e imprimir la información obtenida.\\
	
	Actualmente no le encuentro una gran utilidad, pero normalmente lo usaría para hacer filtros en formularios, contar los caracteres ingresados y rechazar o aceptar el dato dependiendo de las exigencias. Por ejemplo si en un formulario se pide un numero celular, contar y asegurar que sean números y que sean 10 caracteres.
	
	\pagebreak
	
\end{document}