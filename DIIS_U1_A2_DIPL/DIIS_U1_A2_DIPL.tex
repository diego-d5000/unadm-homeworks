\documentclass[spanish,12pt,letterpapper]{article}
\usepackage{babel}
\usepackage[utf8]{inputenc}
\usepackage{graphicx}
\usepackage{hyperref}
\begin{document}
	\begin{titlepage}
		\begin{center}
			\includegraphics[width=0.6\textwidth]{../logoUnADM}~\\[1cm] 
			\textsc{Universidad Abierta y a Distancia de México}\\[0.8cm]
			\textsc{Desarrollo de Software}\\[1.8cm]
			
			\textbf{ \Large Actividad 2. Métodos de desarrollo de software}\\[3cm]
			
			Diego Antonio Plascencia Lara\\ ES1421004131 \\[0.4cm]
			Facilitador(a): Martín Antonio Santos Romero\\
			Asignatura: Introducción a la Ingeniería de Software\\
			Grupo: DS-DIIS-1601-B2-006 \\
			Unidad: I \\
			
			\vfill México D.F\\{\today}
			
		\end{center}
	\end{titlepage}
	
	\section{Metodologías de Desarrollo de Software}
	Las Metodologías de desarrollo de software o modelos de procesos del software son especificaciones abstractas de los pasos y secuencias que se deben seguir para el desarrollo de software, es un marco de trabajo que especifica el proceso a seguir pero no las actividades, permitiendo extender y adaptar el modelo.\\
	
	Hay diferentes modelos pero la mayoría incluye 4 actividades fundamentales para el desarrollo de Software:
	
	\begin{itemize}
	\item \textbf{Especificación.} Donde se define la funcionalidad
	\item \textbf{Diseño e implementación.} Se produce el software que cubra las especificaciones
	\item \textbf{Validación.} Se verifica y asegura que el software es lo que el cliente quiere
	\item \textbf{Evolución.} El software cambia conforme a los cambios de las necesidades del cliente
    \end{itemize}
    	 
	\subsection{Cascada}
	Este modelo es el mas antiguo y representa el principal proceso de desarrollo de software, partiendo las 4 principales actividades en las fases de especificación de requerimientos, diseño de software, implementación y pruebas.\\ 
	
	\textbf{Características:}
	\begin{itemize}
	\item Es convencional, linear y secuencial.
	\item Enfatiza en la planeación, calendarización y tiempos de entrega.
	\item Busca la implementación del sistema en un solo tiempo.
	\item El control del proyecto se mantiene por medio de extensa documentación y las revisiones.
	\item Las revisiones son echas por el usuario y los administrativos de tecnologías de la información entre cada fase, antes de pasar a la siguiente.
	\item En cada fase se definen bien los entregables.
	\end{itemize}
	
	\subsection{Iterativo}
	El modelo iterativo o cascada iterativo es un modelo en respuesta a el cascada original, este promete ser mas flexible y rápido.\\
	
	\textbf{Características:}
	\begin{itemize}
	\item El proyecto se divide en partes pequeñas, permitiendo obtener resultados y retroalimentación mas rápido
	\item Cada iteración es un pequeño proceso tipo cascada, con retroalimentación entre cada fase.
	\item No es fácil de manejar
	\item No hay entregables claros.
	\end{itemize}
	
	\subsection{Prototipado}
	Se enfoca en la creación de prototipo, versiones incompletas del software que esta siendo desarrollado.\\
	
	\textbf{Características:}
	\begin{itemize}
	\item No es una metodología completa o autónoma, es un enfoque para manejar diferentes partes de otra metodología mas completa.
	\item reduce los riesgos del proyecto, dividiéndolo en segmentos mas pequeños para facilitar cambios.
	\item El usuario esta envuelto en el proceso de desarrollo, para aumentar la probabilidad de la aceptación del resultado final.
	\item Se desarrollan pequeños mock-ups del sistema con modificaciones iterativas antes de que el prototipo evolucione para emparejarse con los requerimientos del usuario.
	\item El rápido prototipado ayuda a la visualización de como se vería el sistema final, alentando a el usuario a tener participación activa.
	\item Difícil administración del proyecto. 
	\item No apto para proyectos grandes.
	\item Se corre el riesgo de entregar sistemas no terminados.
	\end{itemize}
	
	\subsection{Espiral}
	Este modelo también puede ser usado por y en conjunto con otros. Combina los elementos del prototipado.\\
	
	\textbf{Características:}
	\begin{itemize}
	\item Se enfoca en analizar y minimizar riesgos.
	\item Secciona el proyecto en segmentos mas pequeños y provee mas facilidad de cambios.
	\item Evaluá los riesgos y la viabilidad de continuar con el proyecto.
	\item Cada ``viaje'' en el espiral pasa por 4 etapas (determinación de objetivos y alternativas; identificar y evaluar alternativas y riesgos; desarrollo y verificación de entregables; planeación de la siguiente iteración)
	\item Cada ciclo comienza con las condiciones de los interesados y termina con la revisión y compromisos.
	\item Todo el tiempo se obtiene un nuevo prototipo para ser evaluado por el cliente
	\item Los errores solo son eliminados en las fases tempranas.
	\end{itemize}
	
	\subsection{Agil}
	En febrero de 2001 tras una reunión entre expertos de la industria del software, sobre como adecuar las metodologías a las necesidades de la industria pues en esa reunión estaban representantes de las metodologías SCRUM, DSDM, ASD, Crystal y FDD que ya se usaban en ese entonces, como alternativa a las metodologías tradicionales, como resultado se creo el manifiesto ágil y la organización ``The Agile Alliance''.\\
	
	\textbf{Características:}
	\begin{itemize}
	\item La prioridad es la satisfacción del cliente con entregas tempranas y continuas que aporten valor.
	\item Cabida a los cambios repentinos para poder ofrecer ventajas competitivas a el cliente.
	\item Entregas frecuentes funcionales con el menor lapso de tiempo posible.
	\item Las personas del negocio y desarrolladores trabajan juntos diariamente.
	\item Confianza, entorno y apoyo necesario a las personas involucradas en el proyecto.
	\item Diálogos cara a cara.
	\item El software que funciona es la principal medida del progreso.
	\item El desarrollo es sostenido, usuarios y desarrolladores deben poder mantener un ritmo de trabajo constante indefinido.
	\item Atención continua a la calidad técnica y al diseño.
	\item La simplicidad es esencial.
	\item Las mejores arquitecturas, requisitos, y diseños rugen de equipos auto-organizados.
	\item El equipo debe reflexionar de como llegar a ser mas efectivo.
	\end{itemize}
	
	\section{MindMap}
	\begin{center}
		\includegraphics[width=1\textwidth]{./mindmap}~\\[1cm] 
\end{center}		
		
	\section{Conclusiones}
	
    \textbf{¿Cuáles son las principales diferencias entre los métodos de desarrollo de software?}
    La diferencia que esta marcada es la evolución y resolución de problemas en y para el tiempo y de necesidades económicas y organizaciones, es decir que las metodologías muestran una evolución.\\
    
    Una metodología resuelve ``los problemas'' de otra anterior, estos problemas se dan al momento de que una organización cambia sus necesidades, por ejemplo la necesidad de reducir costos y producir software mas rápido al ser cada vez mas demandada la tecnología, por esto se puede notar la diferencia entre la metodología de cascada a las ágiles sobre la administración de la cantidad de tiempo en las entregas y en general.\\
    
    \textbf{¿Los métodos ágiles desplazaron a los tradicionales y éstos desaparecieron?}
    No, pues hay metodologías que aún se siguen usando, puesto que no todas la metodologías ágiles se pueden aplicar a grandes empresas o para todos los objetivos, también puede variar y ser condicionado el uso de una metodología u otra por el tamaño del equipo de desarrollo, de igual manera es cierto que las metodologías tradicionales prometen mas calidad en el producto, es útil cuando no se puede errar, por ejemplo cuando la vida de un ser humano esta en riesgo si se comete un error (software medico).\\
    
    \textbf{¿Cuáles son los mejores y por qué?} 
    De nuevo, depende mucho de los objetivos y tamaño de la organización, no hay exactamente una metodología perfecta y la mejor es la adaptada, es decir la que se crea o modifica a partir de otras para las necesidades de quien o quienes la usan.
	
	\pagebreak
	\begin{thebibliography}{9}
	\bibitem{CompSciMag}  Maheshwari, Shikha. 
		\emph{A Comparative Analysis of Different types of Models in Software Development Life Cycle} (2012). International Journal of Advanced Research in Computer Science and Software Engineering. [Disponible en: http://www.ijarcsse.com/docs/papers/May2012/Volum2\_issue5/V2I500405.pdf ]
		
		\bibitem{sommerville}  Sommerville, Ian. 
		\emph{Software engineering} (2011). USA:  Pearson Education. 9th ed. 
		
		\bibitem{sommerville}  Fernández, Jorge. 
		\emph{Introducción a las metodologías ágiles: Otras formas de analizar y desarrolla}. FUOC. Fundación para la Universitat Oberta de Catalunya. 
		
	\end{thebibliography}
	
\end{document}