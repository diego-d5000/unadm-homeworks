\documentclass[spanish,12pt,letterpapper]{article}
\usepackage{babel}
\usepackage[utf8]{inputenc}
\usepackage{graphicx}
\usepackage{hyperref}
\begin{document}
	\begin{titlepage}
		\begin{center}
			\includegraphics[width=0.6\textwidth]{../logoUnADM}~\\[1cm] 
			\textsc{Universidad Abierta y a Distancia de México}\\[0.8cm]
			\textsc{Desarrollo de Software}\\[1.8cm]
			
			\textbf{ \Large Actividad 2. Métodos de desarrollo de software}\\[3cm]
			
			Diego Antonio Plascencia Lara\\ ES1421004131 \\[0.4cm]
			Facilitador(a): Martín Antonio Santos Romero\\
			Asignatura: Introducción a la Ingeniería de Software\\
			Grupo: DS-DIIS-1601-B2-006 \\
			Unidad: I \\
			
			\vfill México D.F\\{\today}
			
		\end{center}
	\end{titlepage}
	
	\section{Metodologías de Desarrollo de Software}
	Las Metodologías de desarrollo de software o modelos de procesos del software son especificaciones abstractas de los pasos y secuencias que se deben seguir para el desarrollo de software, es un marco de trabajo que especifica el proceso a seguir pero no las actividades, permitiendo extender y adaptar el modelo.\\
	
	Hay diferentes modelos pero la mayoría incluye 4 actividades fundamentales para el desarrollo de Software:
	
	\begin{itemize}
	\item \textbf{Especificación.} Donde se define la funcionalidad
	\item \textbf{Diseño e implementación.} Se produce el software que cubra las especificaciones
	\item \textbf{Validación.} Se verifica y asegura que el software es lo que el cliente quiere
	\item \textbf{Evolución.} El software cambia conforme a los cambios de las necesidades del cliente
    \end{itemize}
    	 
	\subsection{Cascada}
	Este modelo es el mas antiguo y representa el principal proceso de desarrollo de software, partiendo las 4 principales actividades en las fases de especificación de requerimientos, diseño de software, implementación y pruebas.\\ 
	
	\textbf{Características:}
	\begin{itemize}
	\item Es convencional, linear y secuencial.
	\item Enfatiza en la planeación, calendarización y tiempos de entrega.
	\item Busca la implementación del sistema en un solo tiempo.
	\item El control del proyecto se mantiene por medio de extensa documentación y las revisiones.
	\item Las revisiones son echas por el usuario y los administrativos de tecnologías de la información entre cada fase, antes de pasar a la siguiente.
	\item En cada fase se definen bien los entregables.
	\end{itemize}
	
	\subsection{Iterativo}
	El modelo iterativo o cascada iterativo es un modelo en respuesta a el cascada original, este promete ser mas flexible y rápido.
	
	\textbf{Características:}
	\begin{itemize}
	\item El proyecto se divide en partes pequeñas, permitiendo obtener resultados y retroalimentación mas rápido
	\item Cada iteración es un pequeño proceso tipo cascada, con retroalimentación entre cada fase.
	\item No es fácil de manejar
	\item No hay entregables claros.
	\end{itemize}
	
	\subsection{Prototipado}
	Se enfoca en la creación de prototipo, versiones incompletas del software que esta siendo desarrollado.
	
	\textbf{Características:}
	\begin{itemize}
	\item No es una metodología completa o autónoma, es un enfoque para manejar diferentes partes de otra metodología mas completa.
	\item reduce los riesgos del proyecto, dividiéndolo en segmentos mas pequeños para facilitar cambios.
	\item El usuario esta envuelto en el proceso de desarrollo, para aumentar la probabilidad de la aceptación del resultado final.
	\item Se desarrollan pequeños mock-ups del sistema con modificaciones iterativas antes de que el prototipo evolucione para emparejarse con los requerimientos del usuario.
	\item El rápido prototipado ayuda a la visualización de como se vería el sistema final, alentando a el usuario a tener participación activa.
	\item Difícil administración del proyecto. 
	\item No apto para proyectos grandes.
	\item Se corre el riesgo de entregar sistemas no terminados.
	\end{itemize}
	
	\subsection{Espiral}
	Este modelo también puede ser usado por y en conjunto con otros. Combina los elementos del prototipado.
	
	\textbf{Características:}
	\begin{itemize}
	\item 
	\end{itemize}
	
	\subsection{V}
	\subsection{Agil}
	\subsection{RAD}
	
	\section{MindMap}
	
	\section{Conclusiones}
	
    \textbf{¿Cuáles son las principales diferencias entre los métodos de desarrollo de software?}
    
    \textbf{¿Los métodos ágiles desplazaron a los tradicionales y éstos desaparecieron?}
    
    \textbf{¿Cuáles son los mejores y por qué?} 
	
	\pagebreak
	\begin{thebibliography}{9}
	\bibitem{CompSciMag}  Maheshwari, Shikha. 
		\emph{A Comparative Analysis of Different types of Models in Software Development Life Cycle} (2012). International Journal of Advanced Research in Computer Science and Software Engineering. [Disponible en: http://www.ijarcsse.com/docs/papers/May2012/Volum2\_issue5/V2I500405.pdf ]
		\bibitem{sommerville}  Sommerville, Ian. 
		\emph{Software engineering} (2011). USA:  Pearson Education. 9th ed. 
		
	\end{thebibliography}
	
	
	
	
\end{document}