\documentclass[spanish,12pt,letterpapper]{article}
\usepackage{babel}
\usepackage[utf8]{inputenc}
\usepackage{graphicx}
\usepackage{hyperref}
\begin{document}
	\begin{titlepage}
		\begin{center}
			\includegraphics[width=0.6\textwidth]{../logoUnADM}~\\[1cm] 
			\textsc{Universidad Abierta y a Distancia de México}\\[0.8cm]
			\textsc{Desarrollo de Software}\\[1.8cm]
			
			\textbf{ \Large Actividad 1. Flujo de entrada y salida}\\[3cm]
			
			Diego Antonio Plascencia Lara\\ ES1421004131 \\[0.4cm]
			Facilitador(a): LETICIA MARIA ANGELES GONZALEZ\\
			Asignatura: Programación orientada a objetos III\\
			Grupo: DS-DPO3-1502S-B2-002 \\
			Unidad: I \\
			
			\vfill México D.F\\{\today}
			
		\end{center}
	\end{titlepage}
	
	El programa que elegí para esta actividad, es uno realizado anteriormente en la asignatura de Estructura de Datos. El programa consiste en un inventario y ``despachador de mesas'' para un restaurante: \\
	
	\url{https://github.com/diego-d5000/tables-system/blob/master/src/edu/diegod/UI/InventoryUI.java} \\
	
	En este programa hay entrada de texto (entrada estándar, teclado) por medio de un form, el cual consta de 5 campos. Estos cinco campos se guardan en un array, para después iterar en el e ir creando un nuevo objeto pasandole el form completo, una vez creado el objeto producto, se agrega a la lista para mostrarlo en una tabla (salida estándar, pantalla). Por lo que el flujo es simple, teclado, programa (se crea objeto con datos), pantalla.\\
	
	
\end{document}