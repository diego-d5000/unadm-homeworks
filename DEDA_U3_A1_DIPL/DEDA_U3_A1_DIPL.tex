\documentclass[spanish,12pt,letterpapper]{article}
\usepackage{babel}
\usepackage[utf8]{inputenc}
\usepackage{graphicx}
\usepackage{hyperref}
\begin{document}
	\begin{titlepage}
		\begin{center}
			\includegraphics[width=0.6\textwidth]{../logoUnADM}~\\[1cm] 
			\textsc{Universidad Abierta y a Distancia de M\'exico}\\[0.8cm]
			\textsc{Desarrollo de Software}\\[1.8cm]
			
			\textbf{ \Large Actividad 1. Arboles y arboles binarios}\\[3cm]
			
			Diego Antonio Plascencia Lara\\ ES1421004131 \\[0.4cm]
			Facilitador(a): Alejandro Francisco Marquez Fuentes\\
			Asignatura: Estructura de datos\\
			Grupo: DS-DEDA-1502S-B1-002 \\
			Unidad: III \\
			
			\vfill M\'exico D.F\\{\today}
			
		\end{center}
	\end{titlepage}
	
	\section{Contador de caracteres \\}
	\textbf{¿Qué es un árbol?\\}
	
	Es una estructura de datos, esta simula un árbol o nodos conectados, por lo que existe una jerarquía entre los datos. El elemento superior se llama raíz, y a excepción de este, todos los demás elementos tienen un padre y cero o mas elementos hijos.\\
	
	Un ejemplo de como se aplicaría en la vida cotidiana, es en las jerarquías, es decir, en las empresas existe un árbol jerárquico, se se quisiera hacer un sistema con permisos de usuarios, probablemente este tema seria útil.\\
	
	\textbf{¿Qué es un árbol binario?\\}
	
	A diferencia de un árbol simple, el árbol binario, cada elemento solo puede tener 2 elementos hijos (hijo izquierdo e hijo derecho). \\
	
	Un ejemplo puede ser un árbol de decisiones, donde solo hay verdadero o falso y una persona hace esta representación mentalmente para saber que decidir. Se me ocurre en una implementación algo mas de entretenimiento como un juego, que dependiendo de las decisiones, llegue a diferentes destinos.
	
	\section{Conclusiones \\}
	Me parece un tema interesante, dentro del tema de estructura para datos. Es elemental entenderlo, ya que hay programas que dependen completamente de esta estructura, como lo es git y otros controladores de versiones, que se basa en un grafo de estados en un proyecto, probablemente en gran parte del codigo requieran una buena implementación de la estructura.
	
	
	
	\pagebreak
	\begin{thebibliography}{9}
		\bibitem{Goodrich} Goodrich, M. y Tamassia, R. (2010).
		\emph{Estructura de datos y algoritmos en Java}. México: CECSA.
		
		\bibitem{WikiWArbol} Wikipedia. 
		\emph{Árbol (informática)}.  {[} Fecha de consulta: \today {]}. Disponible en: \textless https://es.wikipedia.org/wiki/\%C3\%81rbol\_(inform\%C3\%A1tica) \textgreater
		
		\bibitem{WikiWArbol} Wikipedia. 
		\emph{Árbol binario}.  {[} Fecha de consulta: \today {]}. Disponible en: \textless https://es.wikipedia.org/wiki/\%C3\%81rbol\_binario \textgreater

	\end{thebibliography}
	
\end{document}